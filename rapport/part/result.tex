%%%%%%%%%%%%%%%%%%%%%%%%%%%%%%%%%%%%%%%%%%%%%%%%%%%%%%%%%%%%%%%%%%%%%%%%
% RÉSULTATS ET DISCUSSIONS
%%%%%%%%%%%%%%%%%%%%%%%%%%%%%%%%%%%%%%%%%%%%%%%%%%%%%%%%%%%%%%%%%%%%%%%%

% ======================================================================
\section{Base de données}
% ======================================================================

% ----------------------------------------------------------------------
	\subsection{Le retour client}
% ----------------------------------------------------------------------



% ----------------------------------------------------------------------
	\subsection{Efficacité}
% ----------------------------------------------------------------------

Étude de benchmark avec quelques requêtes régulièrement effectuée. Récupérer un csv avec les résultats et en sortir un joli boxplot avec R générer au moment de la compilation.

Voir fonction \texttt{BENCHMARK} de MySQL :
	\begin{verbatim}
		BENCHMARK(count,expr);
	\end{verbatim}
Le problème est que cette fonction ne permet d'être utilisé que sur des expression (\texttt{expr}) ne retournant qu'une ligne ou qu'une colonne.

% ----------------------------------------------------------------------
	\subsection{Maintenabilité}
% ----------------------------------------------------------------------

La base de données est documenté (voir annexe) donc c'est cool.


% ======================================================================
\section{Algorithmes de comparaison}
% ======================================================================

% ----------------------------------------------------------------------
	\subsection{Efficacité}
% ----------------------------------------------------------------------

		\subsubsection{Temps d'exécution}
% ^^^^^^^^^^^^^^^^^^^^^^^^^^^^^^^^^^^^^^^^^^^^^^^^^^^^^^^^^^^^^^^^^^^^^^

Utilisation de R pour la sortie (boxplot)


		\subsubsection{Utilisation mémoire}
% ^^^^^^^^^^^^^^^^^^^^^^^^^^^^^^^^^^^^^^^^^^^^^^^^^^^^^^^^^^^^^^^^^^^^^^

Utilisation de R pour la sortie (boxplot)


% ----------------------------------------------------------------------
	\subsection{Crédibilité du programme}
% ----------------------------------------------------------------------

Comparaison des vrais positifs, faux négatifs et vrais négatifs, faux positifs.


% ======================================================================
\section{Analyse des résultats}
% ======================================================================

% ----------------------------------------------------------------------
	\subsection{Résultat de la comparaison}
% ----------------------------------------------------------------------


% ----------------------------------------------------------------------
	\subsection{Conclusion sur le \emph{Stack}}
% ----------------------------------------------------------------------
